\documentclass[
	letterpaper, % Paper size, specify a4paper (A4) or letterpaper (US letter)
	10pt, % Default font size, specify 10pt, 11pt or 12pt
]{class}

\usepackage{caption}
\usepackage{soul}
\usepackage{subcaption}

\addbibresource{bibliography.bib} % Bibliography file (located in the same folder as the template)

%----------------------------------------------------------------------------------------
%	REPORT INFORMATION
%----------------------------------------------------------------------------------------

\title{GPU Computing\\Parallel implementation of Dijkstra's Algorithm} % Report title

\author{Tricella Davide 08361A} % Author name(s), add additional authors like: '\& James \textsc{Smith}'

\date{\today} % Date of the report

%----------------------------------------------------------------------------------------

\begin{document}

\maketitle % Insert the title, author and date using the information specified above

\begin{center}
    \begin{tabular}{l r}
        Instructor: Professor \textsc{Grossi Giuliano}
    \end{tabular}
\end{center}

%----------------------------------------------------------------------------------------
%	ABSTRACT
%----------------------------------------------------------------------------------------

\begin{abstract}
    The purpose of this paper is to describe the implementation and benchmarking of various parallel implementations of Dijkstra's Algorithm to solve the shortest path problem.
\end{abstract}

%----------------------------------------------------------------------------------------
%	TOC
%----------------------------------------------------------------------------------------

\tableofcontents

\section{Definitions}\label{definitions} % Labels provide a point for referencing, in this case with \ref{definitions} to refer to this subsection number

\begin{description}
    \item[Dataset] The sample of data used to train the Model
    \item[Label] The expected outcome of the prediction
    \item[Model] The group of algorithms that tries to solve the problem
    \item[Overfitting] When the model is too sensible to changes compared to the dataset
    \item[Vanishing Gradient] When the gradient values becomes progressively smaller until they are insignificant for the process
\end{description}

%----------------------------------------------------------------------------------------
%	INTRODUCTION
%----------------------------------------------------------------------------------------

\section{Introduction}

The following text details an analysis on the performances of various parallel versions of Dijkstra's algorithm, implemented through the CUDA Toolkit for parallel computation.\\

\printbibliography % Output the bibliography

%----------------------------------------------------------------------------------------

\end{document}